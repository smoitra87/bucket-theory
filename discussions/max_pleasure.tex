\documentclass{article}
\usepackage{graphicx}
\usepackage{amsmath,amssymb,amsthm}
\usepackage{color}
\usepackage{fullpage}

\def\xcolorversion{2.00}
\def\xkeyvalversion{1.8}

\usepackage[version=0.96]{pgf}
\usepackage{tikz}
\usetikzlibrary{arrows, backgrounds, positioning, fit, shapes}
\usepackage[latin1]{inputenc}

\usepackage{algpseudocode}
\usepackage{algorithm}

\DeclareMathOperator*{\argmax}{arg\,max}
\DeclareMathOperator*{\argmin}{arg\,min}

\begin{document}

\title{Pleasure under Bucket Theory}

\author{
	Moitra, Subhodeep \\ 
	{\tt subhodeep.moitra@gmail.com}
	\and
	Ramachandran, Rohit\\
	{\tt omfgitsrohit@gmail.com}
}

\maketitle

\begin{abstract}
The end product of bucket theory is to derive pleasure at the Pleasure Caf\'e. An unsatisfying consequence from this theory is that the ``purpose'' of life is merely about ``pleasure''. In this article, we will clarify what it means to \textit{pleasure} and provide a rough mathematical model for the same. We also discuss the other loaded word \textit{purpose} and attempt to show that even though the concept is inherently flawed there might be benefits to thinking under this paradigm.
\end{abstract}

\section{Pleasure}
Bucket theory~\cite{bucket2014} defines three kinds of pleasures $ \mathcal{P} = \{peace, joy, thrill \}$ : 
\begin{enumerate}
\item Peace \& Tranquility $(P_{peace})$ .
\item Joy \& Happiness  $(P_{joy})$ .
\item Thrill \& Ecstacy $(P_{thrill})$
\end{enumerate}

$P_{peace}$ is a long lasting pleasurable state of mind ; $P_{joy}$ is a medium term pleasure that lasts a few days such as when taking a vacation or spending time with family ;  $P_{thrill}$ is much more short term lasting merely a few moments. 

\subsection{Mathematical Model}
We attempt to model the creation and consumption of pleasure using a system of differential equations~\ref{eqn:diffeq}. The variables $V_i(t),\; i\in \mathcal{V}$ correspond to the different value buckets. The value resources are converted into a control currency, $C(t)$. Control is then used to purchase the different pleasure nutrients $P_j(t) ; j\in \mathcal{P}$.

\begin{align}
\frac{dC(t)}{dt}  &= \sum_{i \in \mathcal{V}}\alpha_i V_i(t) - \sum_{j \in \mathcal{P}}\beta_j C(t)\\
\frac{dP_j(t)}{dt}  &= \beta_j C(t) - \sum_{i \in \mathcal{V}}\gamma_{ij}V_j(t) \\
\frac{dV_i(t)}{dt}  &= \sum_{j \in \mathcal{P}}\gamma_{ij}P_j(t) - \alpha_i V_i(t)
\ref{eqn:diffeq}
\end{align}

\begin{figure}
\centering
\begin{tikzpicture}[->,>=stealth',shorten >=1pt,auto,node distance=3cm,thick,
  pleasure node/.style={circle,fill=red!20,draw,font=\sffamily\bfseries},
  blank node/.style={circle,fill=blue!20,fill opacity=0,font=\sffamily\bfseries},
  bucket/.style={draw, fill=blue!20, minimum height=3em, minimum width=4em, cylinder, shape border rotate=90, shape aspect=0.1}]


    \node [bucket] (Vw)                                    {$V_{work}(t)$};
    \node [bucket] (Vf) [below of=Vw]                      {$V_{family}(t)$};

    \node [bucket, minimum height=5em, minimum width=4em, fill=green!20] (C) [right of=Vw, yshift = -15mm]                         {$C(t)$};


    \node [pleasure node] (Pj) [right of=C]                      {$P_{joy}(t)$};
    \node [pleasure node] (Pp) [above of=Pj]                       {$P_{peace}(t)$};
    \node [pleasure node] (Pt)  [below of=Pj] 					 {$P_{thrill}(t)$};

 \path[every node/.style={font=\sffamily\small}]
    (Vw) edge node [below] {$\alpha_w$} (C)
    (Vf) edge node [above]   {$\alpha_f$} (C)

    (C) edge  node [below] {$\beta_p$} (Pp)
     edge node [above]   {$\beta_j$} (Pj)
     edge  node [above] {$\beta_t$} (Pt)

	(Pp) edge [bend right] node [above] {$\gamma_{pw}$} (Vw)
	 edge [bend right] node [above] {$\gamma_{pf}$} (Vf)
	 
	(Pj) edge [bend right] node [above] {$\gamma_{jw}$} (Vw)
	 edge [bend left] node [below] {$\gamma_{jf}$} (Vf)
	 
	(Pt) edge [bend left] node [below] {$\gamma_{tw}$} (Vw)
	 edge [bend left] node [below] {$\gamma_{tf}$} (Vf);

  \begin{pgfonlayer}{background}
    \filldraw [line width=4mm,join=round,black!10]
      (Vw.north  -| Vw.east)  rectangle (Vf.south  -| Vf.west)
      (C.north  -| C.east)  rectangle (C.south  -| C.west)
      (Pp.north  -| Pp.east)  rectangle (Pt.south  -| Pt.west);
//      (w1'.north -| l1'.east) rectangle (w2'.south -| e1'.west);
  \end{pgfonlayer}
\end{tikzpicture}
\caption{A simple instance of Bucket Theory. There are two value buckets $\mathcal{V}=\{work,family\}$, the control bank $C$ and three nutrients at the pleasure Caf\'e $\mathcal{P}=\{peace, joy, thrill\}$} 
\ref{fig:pleasure}
\end{figure}


\end{document}